\subsection{Stack}

\newcommand{\stackframe}[3]{
	\begin{frame}[t]{#1}
		\begin{columns}
			\column[t]{0.25\textwidth}
				\ifthenelse{ \equal{#2}{\empty} }{}
				{
					\includegraphics[width=\linewidth]{images/#2}
				}
			
			\column[t]{0.75\textwidth}
				\begin{block}{}
					#3
				\end{block}
		\end{columns}
	\end{frame}
}



\begin{frame}[fragile]{Speicherdauer von »Dingen«}
	\begin{block}{Block}
		\begin{itemize}
			\item Ein \emph{Block} beginnt mit einer \verb|{| und endet mit der nächsten \verb|}|.
			\item {\footnotesize Ein Block kann mehrere \emph{statements} enthalten -- z.B. Definitionen, Zuweisungen, Funktionsaufrufe usw.}
			\item {\footnotesize Anderer Name: \emph{compound statement}}
			\item {\footnotesize Die geschweiften Klammern bezeichnen \emph{keine} Blöcke bei: Klassen, \verb|namespace|, \verb|enum|}
		\end{itemize}
	\end{block}
	
	\pause
	\vspace{1em}
	
	Definiere ich ein »Ding«, so wird es bis zum Ende des Blocks, in welchem es erzeugt wurde, gespeichert.
	
	\vspace{0.5em}
	
	Diese Art der vorgegebenen Speicherdauer nennt sich \emph{automatic storage duration}.
\end{frame}

\begin{frame}[fragile]{Referenz: Definition von »Dingen«, Auf-den-Stack-legen}
	\verb|TYPE name;| \hspace{1em} bzw. \hspace{1em} \verb|TYPE name(PARAMETER);|
	\begin{itemize}
		\item Stellt sicher, dass es Speicher für das »Ding« vom Typ \verb|TYPE| gibt.
		\item Führt den Namen \verb|name| für das »Ding« ein.
		\item Ruft den Konstruktor des »Dings« auf (und übergibt die PARAMETER).
	\end{itemize}
	
	\vspace{1em}
	
	\verb|TYPE name[STATIC_NUMBER];| -- die Variante mit Parametern \emph{existiert nicht!}
	\begin{itemize}
		\item Stellt sicher, dass es Speicher für \verb|STATIC_NUMBER| »Dinge« gibt.
		\item Führt den Namen \verb|name| ein, dieser ist (fast) identisch zur Adresse des nullten Elements.
		\item Ruft nacheinander für jedes Element den \emph{default}-Konstruktor auf, beginnend beim nullten Element.
	\end{itemize}
\end{frame}

\begin{frame}{Reihenfolge der Definitionen}
	\begin{itemize}
		\item Wenn ich (mehrere) »Dinge« innerhalb eines Blocks definiere, werden diese in der Reihenfolge der Definition auf einen Stapel abgelegt -- den \emph{Stack}.
		\item Das ist nicht wörtlich zu nehmen! Der Stack ist nur gedanklich!
		\item Beim Ende des Blocks werden die »Dinge« wieder vom Stack genommen, und zwar in der Reihenfolge, in welcher sie definiert wurden (Standard, 6.6 2).
	\end{itemize}
	
	\pause
	\vspace{1em}
	
	Man kann dieses Verhalten nicht beeinflussen!
\end{frame}

\begin{frame}[fragile]{Referenz: Vom-Stack-nehmen}
	Für ein »Ding«, welches als \verb|TYPE name| definiert wurde:
	\begin{itemize}
		\item Ruft den Destruktor des »Dings« auf.
		\item Gibt den reservierten Speicher wieder frei.
	\end{itemize}
	
	\vspace{2em}
	
	Für ein »Ding«, welches als \verb|TYPE name[STATIC_NUMBER]| definiert wurde:
	\begin{itemize}
		\item Ruft nacheinander für jedes Element den Destruktor auf, beginnend beim nullten Element.
		\item Gibt den reservierten Speicher wieder frei.
	\end{itemize}
\end{frame}


\stackframe{Am Anfang war die Leere\dots}{\empty}{
	\lstinputlisting[language=C++, linerange=1-1]{cpp-code/stack.cpp}
}

\stackframe{Definition von »Dingen« (1)}{stack_1}{
	\lstinputlisting[language=C++, linerange=1-3]{cpp-code/stack.cpp}
}

\stackframe{Definition von »Dingen« (2)}{stack_2}{
	\lstinputlisting[language=C++, linerange=1-4]{cpp-code/stack.cpp}
}

\stackframe{Definition von »Dingen« (3)}{stack_2_plus}{
	\lstinputlisting[language=C++, linerange=1-5]{cpp-code/stack.cpp}
}

\stackframe{Definition von »Dingen« (4)}{stack_3}{
	\lstinputlisting[language=C++, linerange=1-6]{cpp-code/stack.cpp}
}

\stackframe{Definition von »Dingen« (5)}{stack_4}{
	\lstinputlisting[language=C++, linerange=1-7]{cpp-code/stack.cpp}
}

\stackframe{Definition von »Dingen« (6)}{stack_3}{
	\lstinputlisting[language=C++, linerange=1-8]{cpp-code/stack.cpp}
}

\stackframe{Definition von »Dingen« (7)}{stack_2_plus}{
	\lstinputlisting[language=C++, linerange=1-8]{cpp-code/stack.cpp}
}

\stackframe{Definition von »Dingen« (8)}{stack_2}{
	\lstinputlisting[language=C++, linerange=1-9]{cpp-code/stack.cpp}
}
