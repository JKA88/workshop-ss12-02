\section{Objektorientierte Programmierung}
\begin{frame}
    \frametitle{Structures}
    \begin{block}{structs}
    Ein \texttt{struct} ist eine Zusammenfassung mehrerer Objekte zu einem gr��eren.
    Zum Beispiel k�nnte man ein \texttt{struct} "`Quader"' erstellen, welches drei Flie�kommazahlen beinhaltet.
    \end{block}
    \begin{block}{Instanzen}
    Eine solche \texttt{struct} ist lediglich eine abstrakte Beschreibung des Objekts; man arbeitet schlie�lich mit sogenannten \emph{Instanzen} des Objekts. Beispiel Quader: Die Structure an sich beschreibt das abstrakte Objekt, die Instanz einen konkreten Quader ("`der Quader auf meinem Tisch"').
    \end{block}
\end{frame}
\begin{frame}
    \frametitle{Beispiel f�r eine Structure}
    \vspace{0.7cm}
    \includegraphics[width=15cm]{example_code/box.pdf}
\end{frame}

\begin{frame}
    \frametitle{Klassen}
    \begin{block}{Klassen}
    Eine \texttt{class} ist eine \texttt{struct}, die zus�tzlich zu Daten noch Funktionen enth�lt, die auf diesen Daten operieren.
    \end{block}
    \begin{block}{Konstruktor und Destruktor}
    Eine Klasse hat zwei besondere Funktionen, den Konstruktor und den Destruktor; der Konstruktor wird aufgerufen, wenn eine neue Instanz der Klasse erstellt wird, und der Destruktor, wenn die Instanz wieder gel�scht wird. Der Konstruktor hei�t \texttt{klassenname}, der Destruktor \texttt{\~{}klassenname}.
    \end{block}
\end{frame}
\begin{frame}
    \frametitle{Beispiel f�r eine Klasse}
    \vspace{0.7cm}
    \includegraphics[width=8.7cm]{example_code/box2.pdf}
\end{frame}
\begin{frame}
    \frametitle{Etwas anderes Beispiel f�r eine Klasse}
    \includegraphics[width=10.0cm]{example_code/box3.pdf}
\end{frame}