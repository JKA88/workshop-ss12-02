\subsection{Heap}

\begin{frame}{Dynamic storage duration}
	\begin{itemize}[<+->]
		\item »Dinge« \enquote{auf dem Stack} werden automatisch verwaltet
		\item Wie verwalte ich selbst »Dinge«?
		\item Wie nutze ich sehr große »Dinge«?
	\end{itemize}
	
	\vspace{2em}
	
	\uncover<+->
	{
		Die Lösung: \enquote{dynamische} Speicherdauer bzw. der \emph{Heap}
	}
\end{frame}

\begin{frame}[fragile]{Allokation und Deallokation}
	\begin{itemize}[<+->]
		\item Mit \verb|new TYPE| lege ich ein »Ding« auf dem Heap an.
		\item Der \verb|new|-Ausdruck gibt einen Pointer zurück (\verb|TYPE*|).
		\item Mit \verb|delete POINTER;| lösche ich das »Ding« vom Heap, auf welches der \verb|POINTER| verweist.
	\end{itemize}
	
	\onslide<4->
		\begin{block}{ACHTUNG}
			\begin{itemize}[<+-| alert@+>]
				\item Ohne einen Pointer lässt sich ein »Ding« nicht vom Heap löschen!
				\item \emph{Daran denken}, den Speicher wieder freizugeben, wenn er nicht mehr gebraucht wird!
				\item \emph{Niemals} ein »Ding« mehrmals freigeben!
				\item \emph{Niemals} einen Pointer auf ein freigegebenes »Ding« derefenzieren!
				\item Gute Angewohnheit: Nach \verb|delete p;| sofort \verb|p = 0;|, sorgt für (fast) sicheren Programmabsturz im Falle von \verb|*p|
			\end{itemize}
		\end{block}
\end{frame}

\begin{frame}[fragile]{Arrays auf dem Heap}
	Arrays lassen sich auch \enquote{auf dem Heap anlegen}:
	
	\begin{itemize}[<+->]
		\item Mit \verb|new TYPE[DYNAMIC_NUMBER]| lege \verb|DYNAMIC_NUMBER| »Ding« auf dem Heap an.
		\item Der \verb|new|-Ausdruck gibt einen Pointer auf das nullte Element zurück (\verb|TYPE*|).
		\item Mit \verb|delete[] POINTER;| lösche ich den Array vom Heap, auf welchen der \verb|POINTER| verweist.
	\end{itemize}
\end{frame}


{\refframe
\begin{frame}[fragile]{Referenz: Allokation auf dem Heap}
	\verb|new TYPE;| \hspace{1em} bzw. \hspace{1em} \verb|new TYPE(PARAMETER);|
	\begin{itemize}
		\item Alloziert das »Ding« auf dem Heap (den Speicher!).
		\item Ruft den Konstruktor von TYPE auf (und übergibt die PARAMETER).
		\item Gibt die Adresse des Speichers zurück, als \verb|TYPE*|.
	\end{itemize}
	
	\vspace{2em}
	
	\verb|new TYPE[DYNAMIC_NUMBER];| -- die Variante mit Parametern \emph{existiert nicht!}
	\begin{itemize}
		\item Alloziert \verb|DYNAMIC_NUMBER| »Dinge« auf dem Heap (den Speicher!).
		\item Ruft für jedes Element den \emph{default}-Konstruktor von TYPE auf.
		\item Gibt die Adresse des Speichers des nullten Elements zurück, als \verb|TYPE*|.
	\end{itemize}
\end{frame}
}

{\refframe
\begin{frame}[fragile]{Referenz: Deallokation vom Heap}
	\verb|delete ADDRESS;| -- ADDRESS muss eine Rückgabe von \verb|new| sein, sei hier vom Typ TYPE*
	\begin{itemize}
		\item Ruft den Destruktor vom Objekt auf (\verb|ADDRESS->~TYPE();|)
		\item Dealloziert den Speicher vom Heap.
	\end{itemize}
	
	\vspace{2em}
	
	\verb|delete[] ADDRESS;| -- ADDRESS muss eine Rückgabe von \verb|new [...]| sein
	\begin{itemize}
		\item Ruft nacheinander die Destruktoren der Elementes auf, beginnend mit dem letzten.
		\item Dealloziert den Speicher vom Heap.
	\end{itemize}
\end{frame}
}


\begin{frame}[fragile]{Stack vs. Heap}
	\begin{block}{Stack}
		\begin{itemize}
			\item Schnelle Allokation \& Deallokation {\tiny(z.B. ein einzige Operation für beliebig viel »Dinge«!)}
			\item Direkter, schneller Zugriff
			\item Optimierungen gut und einfach(er) für den Compiler
			\item Erzeugung \& Aufräumen ist automatisch
			\item Begrenzter Speicher (z.B. 1~MB)
			\item Größe des »Dings« und Anzahl muss dem Compiler bekannt sein
		\end{itemize}
	\end{block}
	
	\begin{block}{Heap}
		\begin{itemize}
			\item Für große »Dinge«
			\item Eigene Verwaltung der Speicherdauer
			\item Speicherverwaltung während das Programm läuft
		\end{itemize}
	\end{block}
\end{frame}
